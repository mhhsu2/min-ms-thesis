\chapter{Literature Review}
\label{chap: litrev}

\section{Fatigue damage assessment}
Fatigue damage is a critical issue in engineering due to the concern of safety, and accurate estimation of fatigue damage has been a decades-long study in areas such as remanufacturing, transportation equipment, and structural health monitoring. \Citeauthor{fatigue-review-Santecchia2016} provides an extensive overview of fatigue damage models for metals from various perspectives including linear damage rule, continuum damage mechanics, multi-axial as well as variable amplitude loading, energy-based methods, and stochastic-based approaches. However, none of these can be universally accepted because of the complexity of fatigue damage behaviors in reality \cite{fatigue-review-Santecchia2016}.

\section{NDE of fatigue damage}
In many practical scenarios, non-destructive evaluation/testing has been adopted to quantify fatigue damage by investigating the correlation between measurement data and material deterioration \cite{nde-review-ACHENBACH200013}. Some common NDE techniques for evaluating fatigue damage are infrared thermography \cite{nde-thermo-FAN20121}, holographic interferometry \cite{nde-dic}, microwave \cite{nde-microwave}, ultrasonic testing, magnetic methods \cite{nde-magnetic}, acoustic emission approaches, and electrical resistance methods. While numerous NDE methods are available, each of these techniques has its own characteristics and thus is only sensitive to only one or a few specific applications.

\section{LU and NLU applications}
Among various NDE techniques, LU and NLU has demonstrated its applicability in fatigue damage assessment \cite{nde-lu-fatigue-JOSHI1972577, nde-nlu-fatigue-NAGY1998375,nde-nlu-review-Matlack2014, nde-nlu-fatigue-Cantrell}, defect classification \cite{nde-lu-ml-defect-Sambath2011}, and residual stress measurement \cite{nde-lu-rs-review-RUUD198215, nde-lu-rs-Man1987, nde-lu-rs-dike} in materials. In terms of fatigue damage, \Citeauthor{nde-lu-fatigue-JOSHI1972577} showed that ultrasonic attenuation is an indicator of fatigue damage in experiments performed on aluminum and steel \cite{nde-lu-fatigue-JOSHI1972577}. \Citeauthor{nde-nlu-fatigue-NAGY1998375} introduced an experiment setup to monitor the second-order acoustic-elastic coefficient during the cyclic loading test, and demonstrated that the change in nonlinear parameter, which monotonically increases as a function of number of cycles applied, is substantially more than the corresponding change in linear parameters, wave velocity and attenuation \cite{nde-nlu-fatigue-NAGY1998375}. In recent years, \Citeauthor{nde-nlu-review-Matlack2014} presents a comprehensive review of second harmonic generation (SGM) measurements for the NDE of fatigue, thermal aging, and radiation‐induced damage \cite{nde-nlu-review-Matlack2014}. An analytical model developed by \Citeauthor{nde-nlu-fatigue-Cantrell} used a material nonlinearity parameter $ \beta $ extracted from SGM to quantify the level of dislocation substructures and cracks that evolve during cyclic fatigue of planar slip metals, which presented the potential of using SGM to assess the remaining life of the materia. Nonetheless, practical implementation requires that the loading and environmental conditions of fatigue are given \cite{nde-nlu-fatigue-Cantrell}.

\section{RUL prediction}
\section{Residual stress measurement with ultrasounds}
Residual stresses are often existed in mechanical components and have been recognized as a main factor of fatigue failure \cite{rs-fatigue-WEBSTER2001375}. Residual stress measurement, therefore, has been an active area. A review of recent progress of residual stress measurement from \Citeauthor{nde-rs-review-GUO202154} provides a comparison between a variety of methods such as XRD and ultrasounds.


\section{ML-based NDE}