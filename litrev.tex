\chapter{Literature Review}
\label{chap: litrev}

\section{Fatigue damage assessment}
Fatigue damage is a critical issue in engineering due to the concern of safety, and accurate estimation of fatigue damage has been a decades-long study in areas such as remanufacturing, transportation equipment, and structural health monitoring. \Citeauthor{fatigue-review-Santecchia2016} provides an extensive overview of fatigue damage models for metals from various perspectives including linear damage rule, continuum damage mechanics, multi-axial as well as variable amplitude loading, energy-based methods, and stochastic-based approaches. However, none of these can be universally accepted because of the complexity of fatigue damage behaviors in reality \cite{fatigue-review-Santecchia2016}.

\section{NDE of fatigue damage}
In many practical scenarios, NDE has been adopted to quantify fatigue damage by investigating the correlation between measurement data and material deterioration \cite{nde-review-ACHENBACH200013}. Some common NDE techniques for evaluating fatigue damage are infrared thermography \cite{nde-thermo-FAN20121}, holographic interferometry \cite{nde-dic}, microwave \cite{nde-microwave}, ultrasonic testing \cite{nde-lu-fatigue-JOSHI1972577, nde-nlu-fatigue-NAGY1998375,nde-nlu-review-Matlack2014, nde-nlu-fatigue-Cantrell}, magnetic methods \cite{nde-magnetic}, acoustic emission approaches \cite{nde-ae-CHAI2017101}, and electrical resistance methods \cite{nde-electrical-resistance-SUN2007}. While numerous NDE methods are available, each of these techniques has its own characteristics and thus is only sensitive to only one or a few specific applications. Recently, \Citeauthor{nde-review-WISNER2020} presented a review of NDE in material fatigue and stated that when combined, NDE methods have been shown to improve the robustness of damage detection by complementing each other \cite{nde-review-WISNER2020}.

\section{LU and NLU applications}
Among various NDE techniques, LU and NLU have demonstrated their applicability in fatigue damage assessment \cite{nde-lu-fatigue-JOSHI1972577, nde-nlu-fatigue-NAGY1998375,nde-nlu-review-Matlack2014, nde-nlu-fatigue-Cantrell}, defect classification \cite{nde-lu-ml-defect-Sambath2011,nde-lu-ml-defect-s19194216}, and residual stress measurement \cite{nde-lu-rs-Man1987,nde-lu-rs-TANALA199583,nde-lu-rs-LIU2018178} in materials. In terms of fatigue damage, \Citeauthor{nde-lu-fatigue-JOSHI1972577} showed that ultrasonic attenuation is an indicator of fatigue damage in experiments performed on aluminum and steel \cite{nde-lu-fatigue-JOSHI1972577}. \Citeauthor{nde-nlu-fatigue-NAGY1998375} introduced an experiment setup to monitor the second-order acoustic-elastic coefficient during the cyclic loading test, and demonstrated that the change in a nonlinear parameter, which monotonically increases as a function of the number of cycles applied, is substantially more than the corresponding change in linear parameters, wave velocity and attenuation \cite{nde-nlu-fatigue-NAGY1998375}. \Citeauthor{nde-nlu-review-Matlack2014} presented a comprehensive review of second harmonic generation (SHG) measurements for the NDE of fatigue, thermal aging, and radiation‐induced damage \cite{nde-nlu-review-Matlack2014}. An analytical model developed by \Citeauthor{nde-nlu-fatigue-Cantrell} used a material nonlinearity parameter $ \beta $ extracted from SHG measurements to quantify the level of dislocation substructures and cracks that evolve during cyclic fatigue of planar slip metals, presenting the potential of using SHG measurement to assess the remaining life of the material. However, practical implementation of this method requires that the loading and environmental conditions of fatigue are given \cite{nde-nlu-fatigue-Cantrell}.

\section{ML-based NDE}
In recent years, machine learning has been commonly applied to NDE techniques in the automated recognition of patterns in testing signals and the outcome of interest such as fatigue damage levels, defect types, and quality control. For example, \Citeauthor{feature-selection-SHAO2013550} developed several algorithms for feature selection and parameter tuning in quality monitoring of manufacturing processes \cite{feature-selection-SHAO2013550,quality-control-GUO2016141,NAZIR2021806}. \Citeauthor{nde-ml-thermography-defect-Baumgartl2020} implemented a CNN-based in-situ thermographic monitoring system to identify defects produced during the additive manufacturing process of H13 steel \cite{nde-ml-thermography-defect-Baumgartl2020}. An optical interferometry-based real-time quality prediction system using ANN in laser beam welding was developed by \Citeauthor{nde-ml-interferometry-quality-Stad2020} \cite{nde-ml-interferometry-quality-Stad2020}. \Citeauthor{nde-ml-ae-fatigue-LOUTAS2017522} proposed a framework utilizing AE signals for fatigue damage prognostics in composite materials with hidden semi Markov model and Bayesian neural network \cite{nde-ml-ae-fatigue-LOUTAS2017522}. For ultrasonic testings, various features were engineered through fast Fourier transform, wavelet transform, and statistical methods, and fed into ML models for defect classification in \cite{nde-lu-ml-defect-Sambath2011,nde-lu-ml-defect-s19194216}. ML-based NDE is fast growing and more applications such as RUL estimation and residual stress measurement exist in the literature. 

\section{RUL estimation}
RUL estimation has received broad interests these days in many applications on industrial components such as bearing \cite{rul-nn-bearing-BENALI2015150, rul-cnn-bearing-LI20181, rul-ensemble-bearing}, gear \cite{rul-review-gear}, turbofan \cite{rul-statespace-turbo-battery-Mosallam2016,rul-cnn-turbo-LI20181,rul-rnn-turbo-WU2020241}, and lithium-ion battery \cite{rul-statespace-turbo-battery-Mosallam2016,rul-review-battery-LIPU2018115,rul-gpr-battery-9040661}. However, there are relatively fewer recent RUL works directly studying material fatigue. In the early years, \Citeauthor{rul-statespace-fatigue-RAY1996} presented a nonlinear stochastic model for predicting fatigue life for 2024-T3 aluminum alloy based on extended Kalman filter \cite{rul-statespace-fatigue-RAY1996}. \Citeauthor{rul-statespace-fatigue-PENG2015} proposed a Bayesian updating framework based on a physics-based fatigue crack growth model with crack length estimated from piezoelectric sensor signals by a regressor to perform the lap joint fatigue life prognosis for 2024-T3 aluminum alloy \cite{rul-statespace-fatigue-PENG2015}. \Citeauthor{rul-statespace-fatigue-8819426} recently utilized optical and acoustic NDE techniques accompanied with Kalman filter and particle filter to predict RUL in glass fiber reinforced polymer \cite{rul-statespace-fatigue-8819426}. In the literature, most of the RUL research papers associated with fatigue damage rely on physical fatigue modeling and state-space models, e.g., particle filter method. Since physical models contain assumptions and approximation, the effectiveness of physical models may be limited in complex application scenarios.

 Although lots of data-driven approaches for RUL estimation exist \cite{rul-review-SI20111, rul-review-LEI2018799} especially as deep learning has become increasingly prevalent these years \cite{rul-review-KHAN2018241}, the successful cases generally require a sufficient amount of data for model training \cite{rul-cnn-bearing-LI20181, rul-ensemble-bearing, rul-cnn-turbo-LI20181,rul-rnn-turbo-WU2020241, rul-gpr-battery-9040661}. In the field of estimating residual fatigue life, \Citeauthor{rul-nn-fatigue-ultrasound-LIM2018185} developed a data-driven RUL prognosis technique with an artificial neural network (ANN) and nonlinear ultrasonic measurements for 6061-T6 aluminum \cite{rul-nn-fatigue-ultrasound-LIM2018185}. There are, still, a limited number of data-driven methods available for residual fatigue life estimation. 

 Moreover, we barely found methods that perform RUL prognostics by taking only the measurement at the current time step as input, which is the situation we envision for our application on materials screening processes in the remanufacturing industry. The existing examples using approaches such as Kalman filter \cite{rul-statespace-fatigue-RAY1996}, particle filter \cite{rul-statespace-fatigue-8819426}, and recurrent neural network \cite{rul-rnn-turbo-WU2020241} make predictions based on successive measurements. To tackle the issue of lacking previous observations, one relevant research is \Citeauthor{rul-nn-eol-MAZHAR20071184}'s work on EoL products, where the authors integrated Weibull analysis and an ANN model which takes a single measurement to assess the RUL of components for reuse. Due to the lack of data, nonetheless, synthetic data is needed for training the ANN in this work \cite{rul-nn-eol-MAZHAR20071184}.


\section{Residual stress measurement with ultrasound}
Residual stresses often exist in mechanical components and have been recognized as one of the main factors of fatigue failure \cite{rs-fatigue-WEBSTER2001375}. Residual stress measurement, therefore, has been an active area. A review of recent progress of residual stress measurement from \Citeauthor{nde-rs-review-GUO202154} provides a comparison between a variety of methods from the aspects of resolution, applicable object, and limitations \cite{nde-rs-review-GUO202154}. Specifically for ultrasonic testing, the theory of acoustoelastic effect (the presence of stress in solids causes changes in the speeds of ultrasonic waves) for measurement of residual stress has been studied in \cite{nde-lu-rs-Man1987}. \Citeauthor{nde-lu-rs-TANALA199583} compared ultrasonic velocity measurements with X-ray diffraction in determining residual stress across a steel pipe and an alloy plate, which stated that ultrasonic techniques are more efficient in test volume and the cost of equipment \cite{nde-lu-rs-TANALA199583}. In recent studies, \Citeauthor{nde-lu-rs-LIU2018178} implemented a testing system to analyze the accuracy and feasibility of residual stress measurement in 6063-T4 aluminum alloy by ultrasonic longitudinal critically refracted wave based on acoustoelastic theory \cite{nde-lu-rs-LIU2018178}.
