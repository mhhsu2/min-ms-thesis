\chapter{Literature Review}
\label{chap: litrev}

\section{Fatigue damage assessment}
Fatigue damage is a critical issue in engineering due to the concern of safety, and accurate estimation of fatigue damage has been a decades-long study in areas such as remanufacturing, transportation equipment, and structural health monitoring. \cite{fatigue-review-Santecchia2016} provides an extensive overview of fatigue damage models for metals from various perspectives including linear damage rule, continuum damage mechanics, multi-axial as well as variable amplitude loading, energy-based methods, and stochastic-based approaches. However, none of these can be universally accepted because of the complexity of fatigue damage behaviors in reality.

\section{NDE of fatigue damage}
In many practical scenarios, non-destructive evaluation/testing has been adopted to quantify fatigue damage by investigating the correlation between measurement data and material deterioration \cite{nde-review-ACHENBACH200013}. Some common NDE techniques are infrared thermography \cite{nde-thermo-FAN20121}, holographic interferometry \cite{nde-dic}, microwave \cite{nde-microwave}, ultrasonic testing, magnetic methods \cite{nde-magnetic}, acoustic emission approaches, and electrical resistance methods. While numerous NDE methods are available, each of these techniques has its own characteristics and thus is only sensitive to only one or a few specific applications.

\section{LU and NLU applications}
Among various NDE techniques, LU and NLU has demonstrated its ability to identify damage evolution.

\section{ML-based NDE}
\section{RUL prediction}