\chapter{Literature Review}
\label{chap: litrev}

\section{Fatigue damage assessment}
Fatigue damage is a critical issue in engineering due to the concern of safety, and accurate estimation of fatigue damage has been a decades-long study in areas such as remanufacturing, transportation equipment, and structural health monitoring. \Citeauthor{fatigue-review-Santecchia2016} provides an extensive overview of fatigue damage models for metals from various perspectives including linear damage rule, continuum damage mechanics, multi-axial as well as variable amplitude loading, energy-based methods, and stochastic-based approaches. However, none of these can be universally accepted because of the complexity of fatigue damage behaviors in reality \cite{fatigue-review-Santecchia2016}.

\section{NDE of fatigue damage}
In many practical scenarios, non-destructive evaluation/testing has been adopted to quantify fatigue damage by investigating the correlation between measurement data and material deterioration \cite{nde-review-ACHENBACH200013}. Some common NDE techniques for evaluating fatigue damage are infrared thermography \cite{nde-thermo-FAN20121}, holographic interferometry \cite{nde-dic}, microwave \cite{nde-microwave}, ultrasonic testing, magnetic methods \cite{nde-magnetic}, acoustic emission approaches, and electrical resistance methods. While numerous NDE methods are available, each of these techniques has its own characteristics and thus is only sensitive to only one or a few specific applications.

\section{LU and NLU applications}
Among various NDE techniques, LU and NLU has demonstrated its applicability in fatigue damage assessment \cite{nde-lu-fatigue-JOSHI1972577, nde-nlu-fatigue-NAGY1998375,nde-nlu-review-Matlack2014, nde-nlu-fatigue-Cantrell}, defect classification \cite{nde-lu-ml-defect-Sambath2011}, and residual stress measurement \cite{nde-lu-rs-Man1987,nde-lu-rs-TANALA199583,nde-lu-rs-LIU2018178} in materials. In terms of fatigue damage, \Citeauthor{nde-lu-fatigue-JOSHI1972577} showed that ultrasonic attenuation is an indicator of fatigue damage in experiments performed on aluminum and steel \cite{nde-lu-fatigue-JOSHI1972577}. \Citeauthor{nde-nlu-fatigue-NAGY1998375} introduced an experiment setup to monitor the second-order acoustic-elastic coefficient during the cyclic loading test, and demonstrated that the change in nonlinear parameter, which monotonically increases as a function of number of cycles applied, is substantially more than the corresponding change in linear parameters, wave velocity and attenuation \cite{nde-nlu-fatigue-NAGY1998375}. In recent years, \Citeauthor{nde-nlu-review-Matlack2014} presents a comprehensive review of second harmonic generation (SGM) measurements for the NDE of fatigue, thermal aging, and radiation‐induced damage \cite{nde-nlu-review-Matlack2014}. An analytical model developed by \Citeauthor{nde-nlu-fatigue-Cantrell} used a material nonlinearity parameter $ \beta $ extracted from SGM to quantify the level of dislocation substructures and cracks that evolve during cyclic fatigue of planar slip metals, which presented the potential of using SGM to assess the remaining life of the material. Nonetheless, practical implementation requires that the loading and environmental conditions of fatigue are given \cite{nde-nlu-fatigue-Cantrell}.

\section{RUL prediction}
RUL estimation has received broad interests these days in many applications such as bearing \cite{rul-nn-bearing-BENALI2015150, rul-cnn-bearing-LI20181, rul-ensemble-bearing}, gear \cite{rul-review-gear}, turbofan \cite{rul-statespace-turbo-battery-Mosallam2016,rul-cnn-turbo-LI20181,rul-rnn-turbo-WU2020241}, and lithium-ion battery \cite{rul-statespace-turbo-battery-Mosallam2016,rul-review-battery-LIPU2018115,rul-gpr-battery-9040661}. However, there is relatively less state-of-the-art RUL research directly studying material fatigue. In early years, \Citeauthor{rul-statespace-fatigue-RAY1996} presented a nonlinear stochastic model for predicting fatigue life in 2024-T3 aluminum alloy based on extended Kalman filter \cite{rul-statespace-fatigue-RAY1996}. \Citeauthor{rul-statespace-fatigue-PENG2015} proposed a Bayesian updating framework based on a physics-based fatigue crack growth model with crack length estimated from piezoelectric sensor signal by a regressor to perform the lap joint fatigue life prognosis for 2024-T3 aluminum alloy \cite{rul-statespace-fatigue-PENG2015}. \Citeauthor{rul-statespace-fatigue-8819426} recently utilized optical and acoustic NDE techniques accompanied with Kalman filter and particle filter to predict RUL in glass fiber reinforced polymer \cite{rul-statespace-fatigue-8819426}.

In the literature, most of the RUL research papers associated with fatigue damage rely on physical fatigue modeling and state-space models, e.g., particle filter method. Although lots of data-driven research came out these years as machine learning is becoming prevalent, the successful cases generally require a sufficient amount of data for training deep learning models. Hence, there is much fewer data-driven approaches for RUL prediction in fatigue damage.

The existing approaches, moreover, make predictions based on subsequent measurements. We barely found any method that makes prediction based on only the current measurement, i.e., without knowing the history of a sample, which is what we assumed for our application in remanufacturing. [8] is the most relevant one but it has a larger dataset to train Neural Networks.


\section{Residual stress measurement with ultrasounds}
Residual stresses are often existed in mechanical components and have been recognized as a main factor of fatigue failure \cite{rs-fatigue-WEBSTER2001375}. Residual stress measurement, therefore, has been an active area. A review of recent progress of residual stress measurement from \Citeauthor{nde-rs-review-GUO202154} provides a comparison between a variety of methods from the aspects of resolution, applicable object, and limitations \cite{nde-rs-review-GUO202154}. Specifically for ultrasonic testing, the theory of acoustoelastic effect (the presence of stress in solids causes changes in the speeds of ultrasonic waves) for measurement of residual stress has been studied in \cite{nde-lu-rs-Man1987}. \Citeauthor{nde-lu-rs-TANALA199583} compared ultrasonic velocity measurements with X-ray diffraction in determining residual stress across a steel pipe and a alloy plate, which stated that ultrasonic techniques are more efficient in test volume and the cost of equipment \cite{nde-lu-rs-TANALA199583}. In recent studies, \Citeauthor{nde-lu-rs-LIU2018178} implemented a testing system to analyze the accuracy and feasibility of residual stress measurement in 6063-T4 aluminum alloy by ultrasonic longitudinal critically refracted wave based on acoustoelastic theory \cite{nde-lu-rs-LIU2018178}.


\section{ML-based NDE}