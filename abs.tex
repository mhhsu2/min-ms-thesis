Non-destructive evaluation (NDE) and fatigue damage assessment are crucial for quality control of remanufacturing processes. However, critical challenges exist in the development of NDE techniques for used components: each NDE technology is only sensitive to specific fatigue conditions; analytics methods are lacking for quantitatively measuring accumulated mechanical damage and conducting prognostics in an early fatigue stage. In this paper, we propose a machine learning-based NDE technology by combining the strengths of linear ultrasonic (LU) and non-linear ultrasonic (NLU) testings to detect defects at various length scales. Besides, a RUL estimation framework with hierarchical classifiers and S-N curves for the estimation of fatigue damage levels and the prediction of remaining useful life (RUL) of a recycled part is developed. In addition, regression models for estimating residual stress and full width at half maximum (FWHM) of x-ray diffraction (XRD) peak by ultrasonic testings are investigated. The effectiveness of the proposed methods are demonstrated using life cycle fatigue testing data for 5052-H32 aluminum alloy. This research aims to provide a screening system for end-of-life (EoL) products and lead to an increasing usage of secondary materials for remanufacturing and high‐quality products.