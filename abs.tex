Non-destructive evaluation (NDE) and fatigue damage assessment are crucial for quality control of remanufacturing processes. However, critical challenges exist in the development of NDE techniques for used components: individual NDE technology is only sensitive to specific fatigue conditions; analytics methods are lacking for quantitatively measuring accumulated mechanical damage and conducting prognostics in an early fatigue stage. In this paper, we propose a machine learning-based NDE technology by combining the strengths of linear ultrasonic (LU) and nonlinear ultrasonic (NLU) testings to characterize material properties and flaws at various length scales. Besides, a remaining useful life (RUL) estimation framework with hierarchical classifiers and S-N curves for identifying fatigue damage levels and inferring residual fatigue life of recycled parts is developed. In addition, regression models for estimating residual stress and full width at half maximum (FWHM) of X-ray diffraction (XRD) peak with ultrasonic testings are investigated. The effectiveness of the proposed methods is demonstrated using life cycle fatigue testing data for 5052-H32 aluminum alloy. This research aims to provide a screening system for end-of-life (EoL) products and lead to increasing usage of secondary materials for remanufacturing and high‐quality products.