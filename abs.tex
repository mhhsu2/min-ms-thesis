Non-destructive evaluation (NDE) of fatigue damage in metals is crucial for ensuring high product performance and safety. In remanufacturing, NDE for the incoming recycled metal materials is also essential to maximize the benefits of utilizing such materials. However, critical challenges exist in the development of NDE techniques for used components: an individual NDE technology is only sensitive to specific fatigue conditions; and analytics methods are lacking for quantitatively measuring accumulated mechanical damage and conducting prognostics in an early fatigue stage. In this thesis, we propose a novel machine learning-based NDE technology by combining the strengths of linear ultrasonic (LU) and nonlinear ultrasonic (NLU) methods to characterize material properties and flaws at multiple length scales. Besides, a remaining useful life (RUL) estimation framework with hierarchical classifiers and S-N curves for identifying fatigue damage levels and inferring RUL is developed. In addition, regression models are developed to estimate residual stress and full width at half maximum (FWHM) . The effectiveness of the proposed methods is demonstrated using life cycle fatigue testing data for 5052-H32 aluminum alloy. This research provides a screening system for end-of-life (EoL) products and can lead to increasing usage of secondary materials for remanufacturing and high-quality products.
