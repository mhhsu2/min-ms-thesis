\chapter{Introduction}
\label{chap: intro}
Remanufacturing has increasingly received attention because of its energy-saving, eco-friendly, and cost-efficient characteristics. Remanufactured products has been presented in automotive, aerospace, and industrial machinery industry. For example, in the automotive industry, Ford Motor has been recycled and remanufactured components such as engine, transmission, car body, etc., as a long-term tradition. In short, remanufacturing is a process of returning a used product to at least original performance specification from the customers’ perspective. To achieve this, secondary materials screening for quality control in remanufacturing process by estimating the quantities of interest, e.g., remaining useful life (RUL) and residual stress in incoming recycled end-of-life (EoL) products, becomes an essential step and is crucial in increasing the usage of recycled materials.

In recycled components, material fatigue damage is universally presented and it is one of the most influential factors that determines the RUL of a used product. Material fatigue has resulted in many catastrophic accidents in the history and has been studied for several decades; however, the fatigue damage level is hard to be monitored in real world environments due to the stochastic nature of fatigue behaviors and undetermined loading conditions \cite{fatigue-review-Santecchia2016}, which is a critical issue to be addressed.

To quantitatively study fatigue damage in materials, non-destructive evaluation (NDE) methods have been developed \cite{nde-review-WISNER2020}. NDE, also known as non-destructive testing (NDT), is a technique to evaluate material properties without causing damage to the testing parts. For instance, linear ultrasonic (LU) and nonlinear ultrasonic (NLU) testings send ultrasonic waves which propagate in a material and analyze the response signals to evaluate material degradation. Although there exists a variety of NDE techniques, each of these methods is only sensitive to a few specific fatigue conditions and is limited to detecting defects in certain length scales. Figure \ref{fig: lu nlu length scales} illustrates the detectable length scales of LU and NLU testings, where LU testing is robust at detecting macro-scale defects. In contrast, NLU techniques measure nonlinear material parameters to detect defects which are orders of magnitude smaller than the probing wavelength (in e.g. stainless steels, typically on the order of 1mm).

\begin{figure}[tb]
    \includegraphics[width=\linewidth]{fig/lu_nlu_length_scales.png}
    \caption{Capability of defect detection for LU and NLU testings}
    \label{fig: lu nlu length scales}
\end{figure}

There is a lack of research on the estimation and prognosis of RUL of EoL components in the remanufacturing industry even though RUL estimation has been successfully applied in many industrial components with state-of-the-art machine learning (ML) models, e.g., bearing \cite{rul-nn-bearing-BENALI2015150, rul-cnn-bearing-LI20181, rul-ensemble-bearing}, gear \cite{rul-review-gear}, turbofan engine \cite{rul-statespace-turbo-battery-Mosallam2016,rul-cnn-turbo-LI20181,rul-rnn-turbo-WU2020241}, and lithium-ion battery \cite{rul-statespace-turbo-battery-Mosallam2016,rul-review-battery-LIPU2018115,rul-gpr-battery-9040661}. Unlike the examples in the literature, the possible difficulties for RUL estimation in the recycled components are: 
\begin{enumerate*}[label=\itshape\alph*\upshape)]
    \item Continuous and in-situ measurements are not available, i.e., unable to know the historical measurements of a recycled component.
    \item Environmental noises can affect the performance of in-situ sensors and the built algorithms.
    \item The data in this field is hard to collect and thus is not adequate to build robust data-driven models.
\end{enumerate*}

In this research, we proposed a ML-based NDE methodology for the quantification and prognostics of accumulated fatigue damage in recycled materials. First, we integrate LU and NLU testings to leverage the strength of each individual sensor, which has the potential of estimating different fatigue conditions simultaneously. Second, with multi-output hierarchical classifiers, a RUL estimation framework is developed by predicting the loading condition as well as the percentage of fatigue life that a component has undergone, and then the RUL is estimated from a S-N curve. In the proposed approach, both ultrasonic testings sever as ex-situ measurement methods to predict the loading history of a recycled component. Therefore, without continuous monitoring data of a component from its healthy state, the RUL can be inferred by using the current measurement only. Besides, a residual stress measurement method based on the proposed NDE methodology and ML techniques is also investigated, which shows the potential of being an efficient method in terms of speed and cost.

The proposed research is targeted to metals that are widely used in a number of industry sectors. In this preliminary study, the target material is 5052-H32 aluminum alloy which is widely used in truck and auto industries. Life cycle fatigue testings were conducted in different settings to construct a comprehensive database for fatigue progression in the target material. The fatigued specimens were later examined by ultrasonic as well as X-ray diffraction (XRD) measurement. After that, the proposed NDE methodology and prediction framework were developed and tested on the collection of fatigued samples.

The goal of this research is to not only enable effective materials screening but also provide valuable information for the process optimization and control of downstream remanufacturing processes. As such, an effective NDE method that is applicable on the factory floor will lead to greater material recycling and improved quality of products that are produced using recycled materials.

The remaining of this thesis is organized as follows. Chapter \ref{chap: litrev} provides the literature review of the related topics. Chapter \ref{chap: exper} describes the experimental setup and procedures including the design of experiments, life cycle fatigue testing, ultrasonic and XRD measurements. Chapter \ref{chap: model} introduces the ML model development procedure used in both Chapter \ref{chap: rul} and Chapter \ref{chap: reg} where the proposed RUL estimation framework and residual stress measurement method are presented, respectively, with the case study on our fatigue test dataset. Finally, in Chapter \ref{chap: concl}, the contributions of this thesis are summarized and future research directions are mentioned.

