\chapter{Conclusion and Future Work}
\label{chap: concl}

\section{Conclusion}
An ML-based NDE methodology for assessing the accumulated fatigue damage level in terms of RUL, FWHM of XRD peak, and residual stress in recycled metals is developed, which aims to detect defects at various fatigue stages by combining the LU and NLU measurements and provide an ex-situ approach for the prognosis of useful life. An automatic pipeline is used to generate a pool of engineered features from raw ultrasonic signals, select useful features, validate models, and optimize classifiers and regressors. 

A data-driven RUL estimation framework with the hierarchical classifiers and the statistical S-N curve is presented to bridge the research gap of RUL estimation in EoL products where only discrete measurements are available. The design of the hierarchical classification scheme utilizes the characteristics of the fatigue dataset to predict the 7 combinations of the loading amplitude and the percentage of fatigue life. Then, the use of statistical S-N curves incorporates the stochastic nature of fatigue life into the estimation of RUL. The framework relies on simple learning algorithms and does not require a large amount of data for training and validation, which shows the potential to be quickly adopted for other materials. More importantly, the framework does not need successive observations to conduct accurate prognostics of RUL; instead, one measurement at the time a sample is received is enough to provide RUL estimation, addressing the issue that the historical measurements may not be available for recycled components.

In addition, two regression tasks for predicting residual stress and FWHM using ultrasonic testings achieve high prediction accuracy. The high prediction accuracy obtained using the ultrasonic signals demonstrates the potential to apply LU and NLU testing on measuring residual stress and other fatigue damage indicators in a more cost-effective, faster, and non-destructive way.

We envision the proposed NDE methodology and the prediction framework will equip manufacturers with a responsive screening system for incoming recycled materials, and lead to a significant increase in using recycled materials for remanufacturing and high‐quality products that meet customer expectations.

\section{Future work}
In practice, nevertheless, there exist several limitations for the current work and suggested future research efforts are discussed in the following directions.

\subsection{Collect more data for model development and validation}
More data is needed to make our models generalizable in real-world applications even though we have utilized a cross-validation method, LOGOCV, to achieve good prediction accuracy while retaining the generalizability of the model performance with the limited amount of data. In the current LOGOCV, the group left out in the testing set is the three repeated measurements at one location; however, training data from neighboring locations on the same specimen still has high similarity with the testing group, which causes the potential data leakage in the training and validation phase. As a result, more data could enable a better estimate of model performance by treating each specimen as an individual group in LOGOCV.

True fatigue life, i.e., the number of cycles which a sample fails at, for an interrupted fatigue testing specimen is required to fully justify the proposed RUL estimation framework. Currently, the samples used to generate the predicted RUL were not tested until fracture, which prevents us from quantifying the RUL prediction error. Besides, more fatigue life data will allow us to fit the distribution of fatigue life more precisely and/or to select a distribution family, e.g., Weibull, exponential, and log-normal, that better describes the fatigue life behavior. As a result, improvements in the accuracy and robustness of RUL estimation can be achieved.

Additionally, a finer measurement interval can extend the classification task so that classifiers can classify a sample into finer fatigue levels, or allow us to treat the RUL prediction as a regression task. With this, more advanced models can be researched and better applicability as well as flexibility of the framework for practitioners are expected.

\subsection{Multi-sensor fusion for fatigue damage assessment}
The integration of multiple sensors has the capability of detecting different types of defects and improving the robustness of fatigue damage evaluation. However, this work only studies the combination of the LU and NLU testings. Other NDE techniques such as infrared thermography and acoustic emission can be added to the proposed NDE methodology. By integrating more sensors into the system, prediction performance is expected to be further improved.

Moreover, sensor fusion can be categorized into several levels, e.g., data level, feature level, and decision level. We present a feature-level fusion of LU and NLU signals in this research. Other fusion methods, especially decision-level fusion, are worth investigating because sensor selection can be conducted by discarding sensors that degrade the model performance at different fatigue stages. Therefore, a cost-effective quantitative evaluation of fatigue progression system can be designed.

\subsection{Spatial and temporal modeling for fatigue damage evaluation}

With the measurements being made at multiple spatial locations and at multiple time increments, spatiotemporal modeling/interpolation can be performed and provide a full map containing the fatigue damage information for a specimen in space and time. For example, spatial statistics can be used to relate the correlation between multiple measurement locations \cite{dynamic-sampling-Shao,yuhang,CHEN2021306}. Also, the temporal relation can be modeled if enough data is measured at multiple time points. Having the spatiotemporal map for fatigue evolution can give practitioners a better understanding of the fatigue development and enable better decision-making.