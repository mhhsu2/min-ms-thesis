The accomplishment of this thesis is the combination of the efforts of many people, in addition to myself. First and foremost, I would like to express my deepest gratitude and respect to my advisor, Professor Chenhui Shao, for his consistent support and guidance throughout my graduate study. It has been my pleasure to have this opportunity to be his student and work on this interesting topic. Besides, his kindness and encouragement have helped me to tackle difficulties and explore my career goal in my life.

I thank the REMADE Institute for the financial support and industrial advice for this research. This material is based upon work supported by the U.S. Department of Energy’s Office of Energy Efficiency and Renewable Energy (EERE) under the Advanced Manufacturing Office Award Number DE-EE0007897.

I would like to thank Professor Kathryn Matlack and Changgong Kim for the works and discussions of the ultrasonic testing; thank Professor Jingjing Li as well as Bo Pan for the fatigue testing and in-situ monitoring data. This work would not have been completed without their guidance and support.

I am grateful to be a member of the University of Illinois at Urbana-Champaign. The place and people here have enriched my life. My special thank goes to my colleagues Kuan-Chieh Lu, Heng-Sheng Chang, Keo Wu, and so many others, for supporting me in many ways and making my life at UIUC a wonderful experience.

Finally, but most importantly, I express my thank to my family for their unconditional support and endless love, and my brother for his encouragement, without which none of this would have been possible.