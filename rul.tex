\chapter{Remaining Useful Life Prediction}
\label{chap: rul}

In this chapter, we propose a framework for predicting RUL of EoL products based on the ultrasonic testing. The framework has two parts: \begin{enumerate*}[label=\itshape\alph*\upshape)]
    \item a ML classification task and
    \item a RUL inference procedure based on a S-N curve.
\end{enumerate*}  First, ultrasonic signals are fed into ML classifiers to predict the loading condition and the number of fatigue cycles that a sample has gone through. Second, we estimate RUL from a S-N curve with the predicted loading condition and fatigue cycles.

\section{Problem formulation}
Given the goal of predicting RUL on EoL products, we need to formulate this as a ML problem first. In this section, We discuss possible formulations by considering the characteristics of the fatigue dataset and the impact on the ML system in practice.

\subsection{Dataset}
In this RUL prediction task, the dataset is constructed on the ultrasonic measurements on the interrupted fatigue testing specimens in Table \ref{table: interrupted specimens}. There are 15 specimens and each of these were measured at 9 locations alongside 3 repeated measurements, producing 405 observations in total. Notice that we treat one measurement location in one specimen as a sample in the model training and validation procedure with LOGOCV, as described in Section \ref{sec: model train and val}. Besides, each specimen is tested by a combination from 4 loading amplitudes and 3 fatigue levels (the percentage of fatigue life), which forms the labels of a specimen.

\subsection{Target variables}
Obviously, RUL is directly translated by the percentage of fatigue life that a sample has gone through. The percentage of fatigue life as a continuous target variable is normally treated as a regression task. However, since we only have 3 different percentage of fatigue life in the dataset, which is not ideal for regression modeling, we decided to view the percentage of fatigue life as a discrete variable and the problem becomes a classification task.

Loading amplitude is another target variable to be considered because loading condition affects the mechanism of fatigue damage in a material. For instance, at 33\% fatigue life, a sample undergoes 11.7 kN loading and a sample undergoes 14.7 kN loading could exist different fatigue damages. Hence, we place a label that is a combination of loading amplitude and the percentage of fatigue life on each sample. Table XXX presents the labeled RUL prediction dataset.


\section{Classification model development}
\subsection{Single multi-class classifier}
\subsection{Multi-output classifier}
\subsection{Two-stage classifier}
\subsection{Hierarchical classifier}
\subsection{Evaluation metrics}
\subsection{Results}
\section{RUL estimation with a S-N curve}
\subsection{S-N curve with statistical distributions}
\subsection{Results}
\section{Discussion}