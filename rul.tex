\chapter{Remaining Useful Life Prediction}
\label{chap: rul}

In this chapter, we propose a framework for predicting RUL of EoL products based on the ultrasonic testing. The framework has two parts: \begin{enumerate*}[label=\itshape\alph*\upshape)]
    \item a ML classification task and
    \item a RUL inference procedure based on a S-N curve.
\end{enumerate*}  First, ultrasonic signals are fed into ML classifiers to predict the loading condition and the number of fatigue cycles that a sample has gone through. Second, we estimate RUL from a S-N curve with the predicted loading condition and fatigue cycles.

\section{Problem formulation}
Given the goal of predicting RUL on EoL products, we need to formulate this as a ML problem first. In this section, We discuss possible formulations by considering the characteristics of the fatigue dataset and the impact on the ML system in practice.

\subsection{Dataset}
In this RUL prediction task, the dataset is constructed on the ultrasonic measurements on the interrupted fatigue testing specimens in Table \ref{table: interrupted specimens}. There are 15 specimens and each of these were measured at 9 locations alongside 3 repeated measurements, producing 405 observations in total. Notice that we treat one measurement location in one specimen as a sample in the model training and validation procedure with LOGOCV, as described in Section \ref{sec: model train and val}. Besides, each specimen is tested by a combination from 4 loading amplitudes and 3 fatigue levels (the percentage of fatigue life), which forms the labels of a specimen.

\subsection{Target variables}
Obviously, RUL is directly translated by the percentage of fatigue life that a sample has gone through. The percentage of fatigue life as a continuous target variable is normally treated as a regression task. However, since we only have 3 different percentage of fatigue life in the dataset, which is not ideal for regression modeling, we decided to view the percentage of fatigue life as a discrete variable and the problem becomes a classification task.

Loading amplitude is another target variable to be considered because loading condition affects the mechanism of fatigue damage in a material. For instance, at 33\% fatigue life, a sample undergoes 11.7 kN loading and a sample undergoes 14.7 kN loading could exist different fatigue damages. Hence, we place a label that is a combination of loading amplitude and the percentage of fatigue life on each sample. Table \ref{table: rul dataset} presents the labeled RUL prediction dataset.

\begin{table}[tb]
    \centering
    \caption{Summary of the RUL prediction dataset}
    \label{table: rul dataset}
    \begin{tabularx}{\textwidth}{
      >{\centering\arraybackslash\hsize=0.5\hsize}X|
      >{\centering\arraybackslash\hsize=0.6\hsize}X|
      >{\centering\arraybackslash\hsize=0.6\hsize}X|
      >{\centering\arraybackslash}X
    }\hline
      Specimen ID & Measurement Locations & Number of Repeated Measurements & Label (amplitude, percent of fatigue life)\\
      \hline
          1&\multirow{15}{*}{$1\sim 9$}&\multirow{15}{*}{3}&\multirow{2}{*}{Class 1 (11.7 kN, 33\%)}\\
          2& & & \\
          \cline{1-1}\cline{4-4}
          3& & &\multirow{2}{*}{Class 2 (11.7 kN, 67\%)}\\
          4& & & \\
          \cline{1-1}\cline{4-4}
          5& & &\multirow{2}{*}{Class 3 (12.7 kN, 33\%)}\\
          6& & & \\
          \cline{1-1}\cline{4-4}
          7& & &\multirow{2}{*}{Class 4 (12.7 kN, 67\%)}\\
          8& & & \\
          \cline{1-1}\cline{4-4}
          9& & &\multirow{2}{*}{Class 5 (14.7 kN, 33\%)}\\
          10& & & \\
          \cline{1-1}\cline{4-4}
          11& & &\multirow{2}{*}{Class 6 (14.7 kN, 67\%)}\\
          12& & & \\
          \cline{1-1}\cline{4-4}
          13& & &\multirow{3}{*}{Class 0 (0 kN, 0\%)}\\
          14& & & \\
          15& & & \\\hline
        \end{tabularx}
      \end{table}

\section{Design of classifiers}
In this section, classifiers are designed for predicting the loading condition (amplitude) and the percent of fatigue life that a sample had experienced. Several classifiers are developed and the performance of each of those methods is evaluated based on the model development procedure in Chapter \ref{chap: model}.

\subsection{Multi-class classifier}
A multi-class classifier is trained to classify a sample into one of the 7 classes. Figure XXX shows the inference process of a multi-class classifier. In multi-class problems, the classes are mutually exclusive. For example, class 1 and class 3 have nothing related. Despite we claimed that various loading conditions result in different fatigue behaviors in material, however, some similarities are still existed, e.g., at 33\% fatigue life, samples undergone 11.7 kN and 12.7 kN are expected to be on a similar damage level. This idea can be applied to the commonality in samples at different percents of fatigue life as well. With the assumption about mutual exclusivity, multi-class formulation does not capture these characteristics of the data.

\subsection{Multi-output classifier}
On the other hand, multi-output classification can extract the shared information among loading conditions by predicting the loading amplitude and percentage of fatigue life separately. A multi-output classier outputs multiple labels, where each label is considered as a multi-class classification problem. In this design, we build one classifier for predicting the loading amplitude from input signals; another one for classifying a signal into one of the percentages of fatigue life. Each of the two classifiers are trained separately with the same model development procedure but different target variables. We call these two classifiers loading amplitude classifier (LAC) and fatigue cycles classifier (FCC), respectively. Figure XXX depicts the structure of this multi-output classifier.

\subsection{Two-stage classifier}
\subsection{Hierarchical classifier}
\subsection{Evaluation metrics}
\subsection{Results}
\section{RUL estimation with a S-N curve}
\subsection{S-N curve with statistical distributions}
\subsection{Results}
\section{Discussion}